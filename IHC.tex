\documentclass{beamer}

\usepackage[utf8]{inputenc}
\usepackage[portuguese]{babel}

\usetheme{Pittsburgh}

\title{\textbf{IHC na prática}}
\subtitle{Interface e Usabilidade de software}
\author{Milton Daniel Yama Ribera}
\institute{União das Escolas Superiores de Rondônia - UNIRON}
\date{\tiny{\textit{\today}}}

\begin{document}

	\section{Secção inicial}
	\begin{frame}
		\titlepage
	\end{frame}

	\section{Secção 1}
	\begin{frame} {Introdução a Interação entre Homem e Computador}
		\begin{enumerate}
			\item Adotado em meados dos anos 80
			\item Interação determina o destino de um produto
			\item Diferença entre interação e interface
		\end{enumerate}
	\end{frame}

	\section{Secção 2}
	\begin{frame} {Projetando Interações}
		\begin{enumerate}
			\item Definição de projeto de interação
			\item Atividades para criar um projeto de interação
				\begin{itemize}
					\item Levantamento de requisitos
					\item Desenvolvimento de projetos interativos de acordo com o requisito
					\item Construção de versões interativas
					\item Avaliação do que está sendo construído
				\end{itemize}
		\end{enumerate}
	\end{frame}

	\section{Secção 3}
	\begin{frame} {Modelos Conceituais}
		\begin{enumerate}
			\item O que é um modelo conceitual ?
			\item Estratégias para criação de um modelo conceitual
				\begin{itemize}
					\item Visualizar o produto proposto com base nas necessidades do usuário
					\item Definir o melhor modo de estilo de interação
					\item Comportamento da interface
				\end{itemize}
		\end{enumerate}
	\end{frame}
	
	\section{Secção 4}
	\begin{frame} {Usabilidade}
		\begin{enumerate}
			\item Jakob Nielsen
			\item ISO 9126
			\item ISO 9241-11
		\end{enumerate}
	\end{frame}

\end{document}
